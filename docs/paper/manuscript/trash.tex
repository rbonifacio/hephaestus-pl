Although these patterns involve scattering and tangling, a compositional approach (e.g., AOP) does not meet the expressiveness requirement given the heterogeneity in granularity. With this transformational approach, 1) configurability is addressed by enabling \hpl{} with an automatic support to generate product instances with different combinations of assets; 2) extensibility is guaranteed by \hpl{}'s architectural design allowing support to variability management of different assets, independence between the assets and mostly no impact on kernel when inserting new assets in \hpl.

Consequently, we decided to provide extensibility by means of
metaprograms on top of object programs in standard Haskell (Haskell~98
in the case of \hpl). The approach effectively supports extensible
data types (both sums and products) and extensible functions (both
adding equations and extending equations for the relevant idioms).
The metaprogramming approach does not only provide the necessary
extensibility, it can also be easily used with a SPL-like approach for
variable management because the metaprogramming transformations can be
represented by configuration knowledge. Finally, the
metaprogramming approach also enables bootstrapping of \hpl.