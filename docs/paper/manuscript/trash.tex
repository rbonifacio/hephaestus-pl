Consequently, we decided to provide extensibility by means of
metaprograms on top of object programs in standard Haskell (Haskell~98
in the case of \hpl). The approach effectively supports extensible
data types (both sums and products) and extensible functions (both
adding equations and extending equations for the relevant idioms).
The metaprogramming approach does not only provide the necessary
extensibility, it can also be easily used with a SPL-like approach for
variable management because the metaprogramming transformations can be
represented by configuration knowledge. \hpl{} is modularized in a way
that metaprogramming only affects those few abstractions that require
configurability and extensibility. Most assets are separately
compilable and will not be affected by metaprograms.  Finally, the
metaprogramming approach also enables bootstrapping of \hpl.
